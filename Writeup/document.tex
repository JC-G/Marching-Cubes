\documentclass{article}
\title{Real-Time Procedural Terrain Generation with Marching Cubes}
\author{Joseph Chambers-Graham}
\date{}

\setcounter{tocdepth}{3}

\begin{document}
\maketitle
\newpage

\tableofcontents
\section{Introduction}
- project and report overview
\section{Background}
\subsection{Signed Distance Functions}
- what is a SDF\\
- operations on SDF\\
- conventions (what sign is in/out?)\\
- aside: ``density'' functions? An nvidia paper says this, but everyone else says distance, which makes more sense\\

\subsubsection{Noise}
- brief explanation of noise for terrain generation, how to fit it into a SDF
\subsubsection{Approximation}
- not all of the SDFs are exact distances - particularly noise - but good enough for rendering\\
- is it good enough for physics?
\subsection{Marching Cubes Algorithm}
- explain algorithm - basically just a reference?

\section{Design}
\subsection{GPU Implementation}
- How is the algorithm modified\\
- speed comparison (GPU is much faster)
\subsection{LOD system}
 - cannot render everything at max resolution
\subsubsection{Octree}
 - Chunks with same point count, but scaled based on depth of octree

\subsection{Transvoxel Algorithm}
\subsubsection{The problem it solves}
 - cracks in between chunks at different LOD
\subsubsection{How the algorithm works}
 - basically just a reference again\\
 - reference implementation is complicated so could do with explanation
\subsubsection{Adaption of the algorithm to GPU}
 - optimizations to make use of sequential generation have actually been removed\\
 - regardless, this is still pretty fast on the GPU\\
 - similar approach to parallel marching cubes\\
\subsubsection{More issues}
- zero width triangles\\
- edge cases where it still goes wrong (maybe a way to fix this by revisiting the octree)\\

\subsection{Terrain Modification}
\subsubsection{Strategy}
 - adding or removing primitives using SDF operations\\
 - regenerating chunks\\
 - when to regenerate chunks, and which ones\\
 - how many primitives is reasonable before slowdown due to SDF evaluation?\\
 - bounding boxes of primitives to increase this limit - dont need to evaluate parts not affected\\
 
 \subsection{Physics}
 - Just use bullet physics - well-known and optimized library\\
 - zero-width triangles havent turned out to be a problem so far\\
 - generating physics meshes is slow - need to find a solution to this\\
 \subsubsection{SDF-based physics}
 - perhaps look at this - might be a struggle because of non-exact SDFs, and how would we do it for anything that isn't a sphere anyway?
 
 \section{Implementation}
  - C++ with fairly low-level opengl libraries\\
  - key code probably in design section already\\
  - section for the ``boring bits'' of the implementation, rather than key ideas\\
  - perhaps section to explain other things that needed bugfixes?\\
  - something tying it all together?
  
\section{Conclusion}
 - demonstration of final product\\
 - conclusion\\
 - somehting about future work?

%\paragraph{paragraph}
%\subparagraph{subparagraph}
\end{document}